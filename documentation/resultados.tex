\input texbase

\titulo{Exercício Programa 3 - Resultados dos Testes}
\materia{MAC0463/5743 - Computação Móvel}

\aluno{Diogo Haruki Kykuta}{6879613}
\aluno{Fernando Omar Aluani}{6797226}

\begin{document}
\cabecalho

\section{Resultados}
Apresentamos aqui estão os resultados das simulações dos quatro cenários para todos os protocolos de roteamento.
Organizamos eles em tabelas para cada cenário, mostrando os seguintes valores, em ordem, para cada atributo
presente nos relatórios que o \emph{The One} gera:
\begin{itemize}
  \item O valor mínimo desse atributo entre os protocolos, dizendo o nome do protocolo correspondente.
  \item O mesmo que o acima, mas para o valor máximo.
  \item A média dos valores desse atributo entre os relatórios.
  \item O desvio padrão dos valores desse atributo entre os relatórios.
\end{itemize}

Lembrando que em alguns relatórios existems valores $NaN$ para alguns atributos (como por exemplo, o tempo de respostas
em protocolos que não criam respostas para as mensagens), e tais valores foram ignorados no cálculo dessas estatísticas.

\section{CentroSP}

\begin{tabular}{c|c c c c}
  latency\_avg & 88.2166 (Prophet) & 3702.5553 (FirstContact) & 1481.74968 & 1495.76440119 \\
  delivered & 893.0 (Epidemic) & 1410.0 (SprayAndWait) & 1170.4 & 164.997696954 \\
  hopcount\_avg & 1.0 (DirectDelivery) & 119.7245 (FirstContact) & 26.3042 & 46.7255640413 \\
  rtt\_med & NaN & NaN & NaN & NaN \\
  relayed & 1201.0 (DirectDelivery) & 6130771.0 (Epidemic) & 1980449.4 & 2478982.44704 \\
  started & 1202.0 (DirectDelivery) & 6139802.0 (Epidemic) & 1982881.6 & 2482326.0865 \\
  buffertime\_avg & 38.6591 (FirstContact) & 17997.5675 (DirectDelivery) & 7309.13424 & 8685.77777066 \\
  created & 1440.0 (DirectDelivery) & 1440.0 (DirectDelivery) & 1440.0 & 0.0 \\
  aborted & 1.0 (DirectDelivery) & 8953.0 (Epidemic) & 2406.8 & 3436.73219207 \\
  buffertime\_med & 2.0 (FirstContact) & 18000.0 (DirectDelivery) & 7207.8 & 8810.16409382 \\
  sim\_time & 86400.0 (DirectDelivery) & 86400.0 (DirectDelivery) & 86400.0 & 0.0 \\
  response\_prob & 0.0 (DirectDelivery) & 0.0 (DirectDelivery) & 0.0 & 0.0 \\
  dropped & 145.0 (FirstContact) & 6112674.0 (Epidemic) & 1932728.8 & 2499651.24264 \\
  overhead\_ratio & 0.0 (DirectDelivery) & 6864.3651 (Epidemic) & 2024.29872 & 2690.62213427 \\
  delivery\_prob & 0.6201 (Epidemic) & 0.9792 (SprayAndWait) & 0.81278 & 0.114604195386 \\
  removed & 0.0 (DirectDelivery) & 196519.0 (FirstContact) & 39303.8 & 78607.6 \\
  rtt\_avg & NaN & NaN & NaN & NaN \\
  hopcount\_med & 1.0 (DirectDelivery) & 123.0 (FirstContact) & 27.0 & 48.0166637742 \\
  latency\_med & 46.0 (Prophet) & 1590.0 (DirectDelivery) & 690.2 & 677.284105823 \\
\end{tabular}


\section{ParqueIbirapuera}

\begin{tabular}{c|c c c c}
  latency\_avg & 44.6416 (MaxProp) & 533.0569 (DirectDelivery) & 301.88755 & 172.543226131 \\
  delivered & 766.0 (DirectDelivery) & 1607.0 (MaxProp) & 1137.66666667 & 256.72986235 \\
  hopcount\_avg & 1.0 (DirectDelivery) & 70.5813 (FirstContact) & 16.3215333333 & 24.4487584603 \\
  rtt\_med & NaN & NaN & NaN & NaN \\
  relayed & 766.0 (DirectDelivery) & 2097237.0 (MaxProp) & 986622.666667 & 945121.963093 \\
  started & 769.0 (DirectDelivery) & 2103173.0 (MaxProp) & 990030.333333 & 948183.444938 \\
  buffertime\_avg & 7.2124 (FirstContact) & 183.4433 (Epidemic) & 89.262375 & 77.1539670186 \\
  created & 1624.0 (DirectDelivery) & 1624.0 (DirectDelivery) & 1624.0 & 0.0 \\
  aborted & 3.0 (DirectDelivery) & 6896.0 (Prophet) & 3360.16666667 & 3072.78339711 \\
  buffertime\_med & 0.2 (FirstContact) & 109.2 (Epidemic) & 45.95 & 46.0014401948 \\
  sim\_time & 3600.1 (DirectDelivery) & 3600.1 (DirectDelivery) & 3600.1 & 0.0 \\
  response\_prob & 0.0 (DirectDelivery) & 0.0 (DirectDelivery) & 0.0 & 0.0 \\
  dropped & 0.0 (DirectDelivery) & 1757166.0 (Prophet) & 579484.5 & 819586.04838 \\
  overhead\_ratio & 0.0 (DirectDelivery) & 1795.2241 (Epidemic) & 831.517916667 & 812.137387435 \\
  delivery\_prob & 0.4717 (DirectDelivery) & 0.9895 (MaxProp) & 0.700533333333 & 0.158075410977 \\
  removed & 0.0 (DirectDelivery) & 2086185.0 (MaxProp) & 369197.5 & 769303.590939 \\
  rtt\_avg & NaN & NaN & NaN & NaN \\
  hopcount\_med & 1.0 (DirectDelivery) & 63.0 (FirstContact) & 14.3333333333 & 21.9215773966 \\
  latency\_med & 27.3 (MaxProp) & 285.0 (DirectDelivery) & 141.533333333 & 90.219928816 \\
\end{tabular}


\section{Parking}

\begin{tabular}{c|c c c c}
  latency\_avg & 23.0581 (Epidemic) & 398.039 (DirectDelivery) & 191.20162 & 143.277520784 \\
  delivered & 231.0 (DirectDelivery) & 241.0 (Epidemic) & 236.6 & 3.92937654088 \\
  hopcount\_avg & 1.0 (DirectDelivery) & 163.7597 (FirstContact) & 36.36592 & 63.8004441409 \\
  rtt\_med & NaN & NaN & NaN & NaN \\
  relayed & 231.0 (DirectDelivery) & 153865.0 (Epidemic) & 68087.2 & 68016.4193836 \\
  started & 231.0 (DirectDelivery) & 154153.0 (Epidemic) & 68221.4 & 68087.2953218 \\
  buffertime\_avg & 1.9124 (DirectDelivery) & 1.9124 (DirectDelivery) & 1.9124 & 0.0 \\
  created & 243.0 (DirectDelivery) & 243.0 (DirectDelivery) & 243.0 & 0.0 \\
  aborted & 0.0 (DirectDelivery) & 257.0 (Epidemic) & 124.4 & 81.4434773324 \\
  buffertime\_med & 2.0 (DirectDelivery) & 2.0 (DirectDelivery) & 2.0 & 0.0 \\
  sim\_time & 7200.0 (DirectDelivery) & 7200.0 (DirectDelivery) & 7200.0 & 0.0 \\
  response\_prob & 0.0 (DirectDelivery) & 0.0 (DirectDelivery) & 0.0 & 0.0 \\
  dropped & 0.0 (DirectDelivery) & 0.0 (DirectDelivery) & 0.0 & 0.0 \\
  overhead\_ratio & 0.0 (DirectDelivery) & 637.444 (Epidemic) & 283.17922 & 282.307896333 \\
  delivery\_prob & 0.9506 (DirectDelivery) & 0.9918 (Epidemic) & 0.97366 & 0.0161969873742 \\
  removed & 0.0 (DirectDelivery) & 39825.0 (FirstContact) & 7965.0 & 15930.0 \\
  rtt\_avg & NaN & NaN & NaN & NaN \\
  hopcount\_med & 1.0 (DirectDelivery) & 117.0 (FirstContact) & 26.8 & 45.2389212957 \\
  latency\_med & 22.0 (Epidemic) & 262.0 (DirectDelivery) & 125.2 & 87.8052390236 \\
\end{tabular}


\section{USP}

\begin{tabular}{c|c c c c}
  latency\_avg & 1379.4184 (Prophet) & 1648.8626 (FirstContact) & 1470.90068333 & 90.4342641804 \\
  delivered & 2999.0 (FirstContact) & 8966.0 (SprayAndWait) & 5914.83333333 & 2210.33937188 \\
  hopcount\_avg & 1.0 (DirectDelivery) & 8.6519 (FirstContact) & 3.34506666667 & 2.45038987419 \\
  rtt\_med & 1217.4 (FirstContact) & 3130.9 (MaxProp) & 2433.7 & 706.744734681 \\
  relayed & 3013.0 (DirectDelivery) & 353485.0 (Epidemic) & 210891.5 & 126577.764175 \\
  started & 3014.0 (DirectDelivery) & 354338.0 (Epidemic) & 211490.333333 & 126884.584733 \\
  buffertime\_avg & 374.0239 (FirstContact) & 3879.7628 (DirectDelivery) & 2198.27053333 & 1120.85311105 \\
  created & 22467.0 (DirectDelivery) & 25126.0 (SprayAndWait) & 24022.5 & 1086.62439846 \\
  aborted & 1.0 (DirectDelivery) & 874.0 (MaxProp) & 598.833333333 & 314.735559196 \\
  buffertime\_med & 87.8 (FirstContact) & 3574.1 (DirectDelivery) & 1920.98333333 & 1202.58914072 \\
  sim\_time & 86400.0 (DirectDelivery) & 86400.0 (DirectDelivery) & 86400.0 & 0.0 \\
  response\_prob & 0.0246 (FirstContact) & 0.1886 (SprayAndWait) & 0.120616666667 & 0.0584497623795 \\
  dropped & 19496.0 (FirstContact) & 370886.0 (Epidemic) & 172287.833333 & 134360.01799 \\
  overhead\_ratio & 0.0 (DirectDelivery) & 61.5892 (FirstContact) & 35.3758 & 22.5849459047 \\
  delivery\_prob & 0.1327 (FirstContact) & 0.3568 (SprayAndWait) & 0.242616666667 & 0.0830614314161 \\
  removed & 0.0 (DirectDelivery) & 187705.0 (FirstContact) & 56506.6666667 & 80599.2441893 \\
  rtt\_avg & 1589.2041 (FirstContact) & 3137.3417 (MaxProp) & 2533.37816667 & 630.717832815 \\
  hopcount\_med & 1.0 (DirectDelivery) & 5.0 (FirstContact) & 2.5 & 1.25830573921 \\
  latency\_med & 1064.3 (DirectDelivery) & 1345.2 (Epidemic) & 1242.3 & 99.2917082809 \\
\end{tabular}

\end{document}
